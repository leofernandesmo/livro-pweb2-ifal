\chapter{Tabelas}

\thispagestyle{empty} 

A tabela \ref{tab:SI-basicas} mostra algumas unidades básicas do SI.

\begin{table}
\begin{minipage}{\hsize}
\begin{center}
	\caption{Algumas unidades básicas do SI.}
	\label{tab:SI-basicas}
	\begin{tabular}{P{0.40\hsize}|P{0.5\hsize}}
		\hline
		\textbf{Grandeza} & \textbf{Unidade} \\
		\hline
		comprimento & metro (\si{\meter}) \\
		\hline
		massa & quilograma (\si{\kilo\gram}) \\
		\hline
		tempo & segundo (\si{\second}) \\
		\hline
		corrente elétrica & ampère (\si{\ampere}) \\
		\hline
		temperatura & kelvin (\si{\kelvin}) \\
		\hline
		quantidade de matéria & mol (\si{\mol}) \\
		\hline
		intensidade luminosa & candela (\si{\candela}) \\
		\hline
	\end{tabular}
	\source{adaptado do livro Física para Ciências Agrárias e Ambientais, de Leonardo Luiz e Castro e Olavo Leopoldino da Silva Filho.}
\end{center}
\end{minipage}
\end{table}


É uma boa ideia usar o pacote ``longtable'' para criar tabelas, \index{tabelas} pois assim uma mesma tabela pode ocupar várias páginas. Dê uma olhada no código da lista de símbolos, pois ela foi feita com esse pacote.
