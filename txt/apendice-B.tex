\chapter{Tutorial Básico de Linhas de Comando no Linux}

Este apêndice oferece um tutorial básico sobre como utilizar a linha de comando no shell do Linux. Ele abrange desde comandos simples para manipulação de arquivos e diretórios até comandos para acessar a rede e usar ferramentas como o cURL para interagir com APIs.

\section{Comandos Básicos de Arquivos e Diretórios}

\subsection{Criando uma Pasta}
Para criar uma nova pasta (diretório), utilize o comando \texttt{mkdir}:
\begin{verbatim}
mkdir nome_da_pasta
\end{verbatim}
Este comando criará um diretório chamado \texttt{nome\_da\_pasta} no local atual.

\subsection{Navegando até uma Pasta}
Para navegar até um diretório específico, use o comando \texttt{cd} (change directory):
\begin{verbatim}
cd nome_da_pasta
\end{verbatim}
Para voltar ao diretório anterior, você pode usar:
\begin{verbatim}
cd ..
\end{verbatim}

\subsection{Criando um Arquivo HTML}
Para criar um novo arquivo HTML, você pode utilizar o comando \texttt{touch}:
\begin{verbatim}
touch index.html
\end{verbatim}
Este comando criará um arquivo vazio chamado \texttt{index.html} no diretório atual.

\subsection{Listando Arquivos e Diretórios}
Para listar todos os arquivos e diretórios no diretório atual, use o comando \texttt{ls}:
\begin{verbatim}
ls
\end{verbatim}
Você pode adicionar opções ao comando \texttt{ls} para listar arquivos com mais detalhes:
\begin{verbatim}
ls -l      # Lista em formato longo, incluindo permissões e tamanhos.
ls -a      # Lista todos os arquivos, incluindo os ocultos (começam com .).
ls -lh     # Lista com tamanhos de arquivo em um formato legível (KB, MB).
\end{verbatim}

\subsection{Procurando Arquivos}
Para procurar arquivos no sistema, use o comando \texttt{find}:
\begin{verbatim}
find /caminho/para/diretorio -name "nome_do_arquivo"
\end{verbatim}
Este comando pesquisa recursivamente a partir do diretório especificado por arquivos que correspondam ao nome fornecido.

\subsection{Verificando o Histórico de Comandos}
Para ver o histórico dos comandos que você executou, utilize o comando \texttt{history}:
\begin{verbatim}
history
\end{verbatim}
Este comando exibe uma lista de todos os comandos que você executou na sessão atual do terminal.

\section{Comandos de Rede e Acesso a Sites Web}

\subsection{Verificando a Configuração de Rede}
Para verificar a configuração de rede da sua máquina, use o comando \texttt{ifconfig} (ou \texttt{ip} nas versões mais recentes do Ubuntu):
\begin{verbatim}
ifconfig        # Exibe as interfaces de rede e suas configurações.
ip addr show    # Comando alternativo mais moderno para mostrar interfaces.
\end{verbatim}

\subsection{Acessando Sites Web}
Para acessar e recuperar dados de um site a partir da linha de comando, você pode usar o comando \texttt{wget}:
\begin{verbatim}
wget https://www.exemplo.com
\end{verbatim}
Este comando baixa o conteúdo da URL especificada para o diretório atual.

\section{Usando o cURL para Acessar APIs}

\subsection{Introdução ao cURL}
O cURL é uma ferramenta de linha de comando que permite realizar requisições a URLs, sendo particularmente útil para acessar APIs RESTful.

\subsection{Realizando uma Requisição GET}
Para realizar uma requisição GET simples a uma API, use o seguinte comando:
\begin{verbatim}
curl https://api.exemplo.com/dados
\end{verbatim}
Este comando envia uma requisição GET para a URL especificada e exibe a resposta no terminal.

\subsection{Enviando Dados com uma Requisição POST}
Para enviar dados em uma requisição POST, você pode usar o cURL da seguinte forma:
\begin{verbatim}
curl -X POST https://api.exemplo.com/usuario \
     -H "Content-Type: application/json" \
     -d '{"nome": "João", "idade": 30}'
\end{verbatim}
Neste exemplo, o comando envia um objeto JSON contendo os dados do usuário para a API.

\subsection{Autenticação com cURL}
Se a API requer autenticação, você pode adicionar um cabeçalho de autorização à sua requisição:
\begin{verbatim}
curl -H "Authorization: Bearer seu_token_aqui" \
     https://api.exemplo.com/dados_protegidos
\end{verbatim}
Este comando envia uma requisição GET autenticada com um token de acesso.

\section{Outros Comandos Úteis}
\begin{itemize}
    \item \textbf{pwd:} Mostra o diretório atual (o caminho completo até onde você está).
    \begin{verbatim}
    pwd
    \end{verbatim}
    
    \item \textbf{rm:} Remove (exclui) arquivos ou diretórios.
    \begin{verbatim}
    rm nome_do_arquivo
    rm -r nome_do_diretorio  # Remove um diretório e todo o seu conteúdo.
    \end{verbatim}
    
    \item \textbf{cp:} Copia arquivos ou diretórios.
    \begin{verbatim}
    cp arquivo_origem arquivo_destino
    cp -r diretorio_origem diretorio_destino  # Copia diretórios recursivamente.
    \end{verbatim}
    
    \item \textbf{mv:} Move ou renomeia arquivos ou diretórios.
    \begin{verbatim}
    mv arquivo_origem arquivo_destino
    mv nome_antigo nome_novo  # Renomeia um arquivo ou diretório.
    \end{verbatim}
    
    \item \textbf{man:} Exibe o manual de um comando, útil para aprender sobre suas opções e uso.
    \begin{verbatim}
    man comando
    \end{verbatim}
\end{itemize}

% Comentário: Este tutorial fornece uma introdução prática para alunos que precisam utilizar o shell do Linux em um ambiente de desenvolvimento web. Comandos adicionais ou mais avançados podem ser incluídos conforme necessário.
